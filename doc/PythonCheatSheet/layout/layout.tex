\documentclass[notumble,10pt,a4paper]{leaflet} 

% notumble - By default (tumble) the contents of the backside sheet is printed upside down. The option no tumble supresses that.

% Please find the remaining options at http://ctan.math.illinois.edu/macros/latex/contrib/leaflet/leaflet-manual.pdf
\usepackage[utf8x]{inputenc}
\usepackage{fontspec}

\usepackage{color}
\usepackage{flowfram}
\usepackage{graphicx}
\usepackage{wrapfig}
\usepackage{hyperref}      % For including urls
\usepackage{rotating}
\usepackage{multirow}
\usepackage{array}
\usepackage{multirow}
\usepackage{titlesec}
\usepackage[usenames,dvipsnames]{xcolor}
\usepackage{setspace}    % Adjust line spacing    
\onehalfspacing          % Adjust line spacing  \doublespacing 
\usepackage[inline]{enumitem}
\usepackage{capt-of}
%\titleformat*{\subsection}{\color{Blue}}

%FONT Change
%\renewcommand{\familydefault}{cmss} 
%To Draw a horizontal Line
\newcommand{\sectionline}{
  \nointerlineskip \vspace{\baselineskip}
  \hspace{\fill}\rule{0.8\linewidth}{.7pt}\hspace{\fill}
  \par\nointerlineskip \vspace{\baselineskip}
}
%\AddToBackground{2}{\includegraphics[width=29.7cm]{bkf}}
%\AddToBackground{2}{\includegraphics[width=29.7cm]{cmy}}

% To create a border along the top of each page
% To change border color just search on internet for cmyk colour codes and change the values in the brackets after the option cmyk

\vtwotonetop{1cm}{0.6\paperwidth}{[cmyk]{0.35,0,0.67,0.41}}{topleft}%
{0.4\paperwidth}{[cmyk]{0.35,0,0.67,0.41}}{topright}

% To create a border along the bottom of each page
% To change border color just search on internet for cmyk colour codes and change the values in the brackets after the option cmyk

\vtwotonebottom{1cm}{0.6\paperwidth}{[cmyk]{0.35,0,0.67,0.41}}{bottomleft}%
{0.4\paperwidth}{[cmyk]{0.35,0,0.67,0.41}}{bottomright}


\usepackage{listings}
\usepackage{xcolor}
\definecolor{codegreen}{rgb}{0,0.6,0}
\definecolor{codegray}{rgb}{0.5,0.5,0.5}
\definecolor{codepurple}{rgb}{0.58,0,0.82}
\definecolor{backcolour}{rgb}{0.95,0.95,0.92}

\lstdefinestyle{mystyle}{
    backgroundcolor=\color{backcolour},   
    commentstyle=\color{codegreen},
    keywordstyle=\color{magenta},
    numberstyle=\tiny\color{codegray},
    stringstyle=\color{codepurple},
    basicstyle=\ttfamily\footnotesize,
    breakatwhitespace=false,         
    breaklines=true,                 
    captionpos=b,                    
    keepspaces=true,                 
    numbers=none,                    
    numbersep=5pt,                  
    showspaces=false,                
    showstringspaces=false,
    showtabs=false,                  
    tabsize=2
}

\lstset{style=mystyle}
